\documentclass[doctor,euler,twoside,openright]{sysuthesis}
\usepackage{xeCJK}%为个别字打不出来准备,或xtex
\usepackage{amsthm,amsmath,amssymb,amsmath, amsfonts,mathrsfs}
\usepackage{fourier}%为希腊字母准备,它可以全局改变希腊字体,其字体也比较好看,但它只限于全局更改和数学模式。
% 默认twoside 双面打印
% 将master修改为bachelor, doctor or master
% 要使用adobe字体,添加adobefonts选项
% 要使用Mac系统的字体,添加macfonts选项
% 使用euler数学字体,如不愿使用,去掉euler
% 使用外文写作,请添加notchinese
\def\pozhehao{\raisebox{0.5mm}{----}}%中文破折号

\renewcommand {\thetable} {\thechapter{}-\arabic{table}}
\renewcommand {\thefigure} {\thechapter{}-\arabic{figure}}
\renewcommand {\theequation} {\thechapter{}-\arabic{equation}}

\newcommand{\tabincell}[2]{\begin{tabular}{@{}#1@{}}#2\end{tabular}}
\allowdisplaybreaks
\newcounter{magicrownumbers}
\usepackage[figuresright]{rotating}
\usepackage{booktabs}
\usepackage{threeparttable}

\usepackage{adjustbox}
\usepackage[graphicx]{realboxes}
\usepackage{rotating}

\usepackage{tikz}%这两行用于顶部径迹室做辅助线
\usetikzlibrary{positioning}

\setCJKmainfont[BoldFont={STHeiti}, ItalicFont={STKaiti}]{STSong}
\setmainfont{Times New Roman}%可要可不要
% 设置图形文件的搜索路径
\graphicspath{{figures/},{figures/template/}}
%\graphicspath{{figures/template/}}
%仅用于本示例文档中显示特殊字符串
\usepackage{xltxtra}
\usepackage{datetime} %日期
\newcommand{\cntoday}{\number\year 年 \number\month 月}% \number\day 日
\newcommand{\entoday}{\ifcase \month \or January\or February\or March\or %
April\or May \or June\or July\or August\or September\or October\or November\or %
December\fi, \number \year}% \number\day 日
%%%%%%%%%%%%%%%%%%%%%%%%%%%%%%
%% 封面部分
%%%%%%%%%%%%%%%%%%%%%%%%%%%%%%

  % 中文封面内容
  \title{中文标题}%一般情况下扉页和封皮、书脊共用一个标题文本,可以不用定义\spinetitle(仅硕博有用), \covertitle(本硕博均有用)和\encovertitle(仅本科有用)。特殊情况见下。
 % \spinetitle{\small{中基线反应堆中微子实验的若干问题研究\raisebox{-3pt}{(Beta)}}}
  %特殊情况1:本例中\title命令里含有换行控制字符,这会导致制作书脊的时候出现错误,例如如果你注释掉\spinetitle{...}这一行就会报错。这时需要定义一个不含换行等命令的\spinetitle,这并不表示\spinetitle里不能有任何命令——只能使用有限的命令。
  %特殊情况2:本例中标题过长,所以需要缩小书脊标题的字号。
  %特殊情况3:本例中中英文混排,由于tex竖排的原理限制,中英文基线不重合,所以需要人工调整英文的基线。具体调整量根据不同字体有所不同。
  %\covertitle{中基线反应堆中微子振荡实验的若干问题研究}
  %\covertitle{中文题目第一行\\中文题目第二行}
  %不要在此调整封皮字体大小! Do not set Cover Page font size here!
  %特殊情况4:本例中\title中含有多个换行,导致标题超过了两行。根据制本厂规定,封皮标题不能超过两行。因此需要定义封皮使用的标题\covertitle. 如果你注释掉这一行,就会发现封皮不符合规定。
  %\encovertitle{Research on medium-baseline reactor neutrino oscillation experiments}
  %\encovertitle{English Title Line 1\\English Title Line 2\\English Title Line 3}
  %不要在此调整封皮字体大小! Do not set Cover Page font size here!
  %特殊情况5:仅本科生有用。本科封皮中有英文标题,不超过三行。与上类似。

  \author{王五}
  \depart{物理系}%系别,硕博请用系代号,本科请用全称如
  %\depart{数理化和信息工程系}
  \major{粒子物理与原子核物理}%专业,硕博请用全称,本科不需要
  \advisor{WW\ 教授}
  \cnfeild{中微子物理}
  %\coadvisor{冯晨珠\ 教授}%第二导师,没有请注释掉
  \studentid{18111046}%For bachelor only
  \submitdate{\cntoday}

  % 英文封面内容
  \entitle{English Title}%Research on related questions of neutrino mass hierarchy with medium-baseline reactor neutrino oscillation experiments
  \enauthor{Wu Wang}
  \enmajor{Particle Physics and Nuclear Physics}
  \enadvisor{Prof. WW}
 \enfeild{Neutrino Physics}
  %\encoadvisor{Prof. Chenzhu Feng}%另外一个导师
  \ensubmitdate{\entoday}
  
%%%%%%%%%%%%%%%%%%%%%%%%%%%%%%%%%%%%%%%%%%%%%%%%%%%%%%%%%%%%%%%%%%%%%
% If you use another language instead of chinese or english, then you
% should define some strings and provide information in your language.
%%%%%%%%%%%%%%%%%%%%%%%%%%%%%%%%%%%%%%%%%%%%%%%%%%%%%%%%%%%%%%%%%%%%%
%  \otherustcstr{zhong guo ke xue ji shu da xue}%A translation of `University of Science and Technology of China' in your language
%  \otherthesisstr{shuo shi xue wei lun wen}%A translation of `A dissertation for doctor(master/bachelor)'s degree' in your language
%  \otherauthorstr{xing ming}%A translation of `Author' in your language
%  \otherdepartmentstr{yuan xi}%A translation of `Department' in your language
%  \otherstudentidstr{xue hao}%A translation of `Student ID' in your language
%  \othersupervisorstr{dao shi}%A translation of `Supervisor' in your language
%  \otherfinishedtimestr{ri qi}%A translation of `Finished Time' in your language
%  \otherspecialitystr{zhuan ye}%A translation of `Speciality' in your language
%  \othertitle{zhong guo ke xue ji shu da xue tong yong xue wen lun wen shi li wen dang}
%  \otherauthor{zhao qian sun}
%  \otheradvisor{zhou wu zheng}
%  \othercoadvisor{feng chen zhu}
%  \othersubmitdate{hou nian ma yue}
%  \othermajor{mou zhuan ye}
%  \otherdepart{mou xi}

\begin{document}

  % 封面
  \maketitle

%特别注意,以下述顺序为准,在对应部分添加文档部件,切勿颠倒顺序:
%本科论文的文档部件顺序是:
%    frontmatter:致谢、目录、中文摘要、英文摘要、
%    mainmatter: 正文章节
%    backmatter: 参考文献或资料注释、附录
%硕博论文的文档部件顺序是:
%    frontmatter:中文摘要、英文摘要、目录、符号说明
%    mainmatter: 正文章节
%    backmatter: 参考文献、附录、致谢、发表论文
%%%%%%%%%%%%%%%%%%%%%%%%%%%%%%
%% 前言部分
%%%%%%%%%%%%%%%%%%%%%%%%%%%%%%
\frontmatter
\makeatletter
\ifustc@bachelor
	%%%%%%%%%%%%%%%%%
	%本科论文修改这里
	%%%%%%%%%%%%%%%%%
	% 致谢
	\include{chapter/thanks}
	
	%目录部分
	%目录
	\tableofcontents
	%默认表格、插图、算法索引名称分别为“表格索引”、“插图索引”和“算法索引”
	%如果需要自行修改lot,lof,loa的名称,请定义
	%\ustclotname{...}
	%\ustclofname{...}
	%\ustcloaname{...}

	% 表格索引
	\ustclot
	% 插图索引
	\ustclof
	%算法索引 
	%如果需要使用算法环境并列出算法索引,请加入补充宏包。
	\ustcloa
	% 摘要
	\include{chapter/abstract}%此文件中含有中英文摘要
\else
	%%%%%%%%%%%%%%%%%
	%硕博论文修改这里
	%%%%%%%%%%%%%%%%%
	% 摘要
	\include{chapter/abstract}%此文件中含有中英文摘要
	% 目录
	\tableofcontents
	%默认表格、插图、算法索引名称分别为“表格索引”、“插图索引”和“算法索引”
	%如果需要自行修改lot,lof,loa的名称,请定义
	%\ustclotname{...}
	%\ustclofname{...}
	%\ustcloaname{...}

	% 表格索引
	\ustclot
	% 插图索引
	\ustclof
	%算法索引 
	%如果需要使用算法环境并列出算法索引,请加入补充宏包。
	%\ustcloa
	
	%符号说明,需要加入补充包
	%\include{chapter/denotation}%不是必需的,如果不想列出请注释掉
\fi
\makeatother

%%%%%%%%%%%%%%%%%%%%%%%%%%%%%%
%% 正文部分
%%%%%%%%%%%%%%%%%%%%%%%%%%%%%%
\mainmatter
\setlength{\baselineskip}{20pt}%行间距20磅
  
\chapter{绪论}
\label{chap:introduction}
\section{中微子简介}
\subsection{中微子的提出}
中微子的提出来自于泡利解释$\beta$衰变中观察到
\subsection{标准模型中的中微子}
直到实验证实了中微子存在振荡行为,违背了标准模型的对中微子零质量的假设预言。此处测试文献引用\cite{Fogli:2007tx,Machado:2011tn}。
\section{中微子振荡}
\section{中微子实验}
\section{论文立意以及结构}
%\subsection{系统要求}

  %\include{chapter/chap-guide}
  %\include{chapter/chap-example}
  
  %自行添加
  %\include{chapter/...}

%%%%%%%%%%%%%%%%%%%%%%%%%%%%%%
%% 附件部分
%%%%%%%%%%%%%%%%%%%%%%%%%%%%%%
\backmatter

  % 附录,没有请注释掉 % 这和参考文献掉了个顺序
  \begin{appendix}
    \include{chapter/chap-req}
  \end{appendix}

  % 参考文献
  % 使用 BibTeX
  % 选择参考文献的排版格式。注意sysubib这个格式不保证完全符合要求,请自行决定是否使用
  \bibliographystyle{sysubib}%{GBT7714-2005NLang-UTF8}
  \bibliography{bib/tex}
  \nocite{*} % for every item
  
  % 不使用默认的 BibTeX
  %\include{chapter/bib}



  \makeatletter
  \ifustc@bachelor\relax\else
    % 致谢
    \include{chapter/pub}
    % 致谢
    \include{chapter/thanks}%硕博致谢部分
    % 中大授权书
    
\chapter{中山大学学位论文版权使用授权书}
\label{chap:authorization}
中山大学学位论文版权使用授权书
本学位论文作者及指导教师完全了解“中山大学硕士、博士(硕士)学位论文版权使用规定”,同意中山大学保留并向国家有关部门或机构送交学位论文的复印件和电子版,允许论文被查阅和借阅。本人授权中山大学可以将本学位论文的全部或部分内容编入有关数据库进行检索,也可采用影印、缩印或扫描等复制手段保存和汇编学位论文。
\vskip1.8cm %1.8cm 2cm 3.1cm
  \begin{center}
  {\xiaosi
   \begin{tabular}{lc}
    作者签名:& \ustc@underline[6cm]{}\\
\\
     导师签名:& \ustc@underline[6cm]{}\\
     \\
     &\ustc@underline[1cm]{}年\ustc@underline[1cm]{}月\ustc@underline[1cm]{}日\\
	  \end{tabular}
	  }
    \end{center}%中大授权书
  \fi
  \makeatother

\end{document}
